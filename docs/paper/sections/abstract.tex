\begin{abstract}

Identifying the concepts covered in a university course based on a high level description is a necessary step in the evaluation of a university's program of study.
To this end, data describing university courses is readily available on the Internet in vast quantities.
However, understanding natural language course descriptions requires manual inspection and, often, implicit knowledge of the subject area.
Additionally, a holistic approach to curricular evaluation involves analysis of the prerequisite structure within a department, specifically the conceptual overlap between courses in a prerequisite chain.
In this work we apply existing topic modeling techniques to sets of course descriptions extracted from publicly available university course catalogs.
%We limit the scope of this work to Computer Science departments in order to focus the number of possible topics and to allow benchmarking against third party expectations published within the field.
The inferred topic models correspond to concepts taught in the described courses.
The inference process is unsupervised and generates topics without the need for manual inspection.
We present an application framework for data ingestion and processing, along with a user-facing web-based application for inferred topic presentation.
The software provides tools to view the inferred topics for a university's courses, quickly compare departments by their topic composition, and visually analyze conceptual overlap in departmental prerequisite structures.
\end{abstract}
