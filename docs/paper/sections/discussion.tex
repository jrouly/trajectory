\section{Discussion}
\label{sec:discussion}

%------------------------------------------------

The results of this study are promising.
Exploratory K-Means clustering resulted in clusters with a high level of completeness.
Even superficial manual analysis of clusters indicated that the collected data were being appropriately grouped.
Similar findings were encountered after the application of \ac{lda} topic modeling.

%------------------------------------------------

Inferred topics generally fall into one of two categories for each course.
The first category of topic includes relevant terms and keywords found in the course description.
A highly specialized course in a particular subfield might contain a set of these topics that relate to keywords from within the specific domain of the course.
The second category of topic includes more generic words common to a large number of courses.
Topics in this category often relate to concepts common across courses, eg.\ student research or exam and project information.

%------------------------------------------------

Take, for example, a computer science course in ethics.
At George Mason University the senior level Computer Science ethics course is CS 306.
The primary topic inferred for this course in a given run of \ac{lda} might look like ``ethic, comput, issu, profession, social, impact, privaci, digit, context, technolog.''
Additionally, the prerequisite course to CS 306, CS 105, shares this same topic.
While the topic itself is a frequency distribution over vocabulary, and does not quantifiably evaluate to an ``ethics'' topic, manual inspection clearly shows that this topic involves ethocs and social issues as they relate to technology professionals.

%------------------------------------------------

Our confidence in the applicability of \ac{lda} as a course content inference system stems from the appearance of the same or related topics within the same or related courses.
Inspecting the same course at different institutions results in the same topics being inferred at each institution.
Inspecting prerequisite courses within the same institution illustrates the relationship between the courses by highlighting conceptual overlap, ie.\ the appearance of the same topic in both course and prerequisite.
As mentioned in \sref{sec:evaluation}, third party course descriptions act as an additional evaluative metric for this approach.
A high level of consistency is indicated by the same topics being inferred for the same course at multiple institutions, related courses within the same institution, and the third party benchmark course.

%------------------------------------------------

