\section{Background on LDA}
\label{sec:background}

%------------------------------------------------

Topic modeling, a form of latent variable modeling, is an unsupervised machine learning method which attempts to recreate the distribution of so-called ``topics'' an author used to generate a corpus of documents.
The term topic is used to describe a frequency distribution of terms within a vocabulary.
In this use, a topic can be understood to represent an academic concept covered within the context of a course.
The topics discovered in a corpus can be used to categorize documents and provide structure to an otherwise unknown dataset.

%------------------------------------------------

\acf{lda} is a specific type of topic modeling which assumes that a mix of multiple topics exist within a single document in some proportion (ie.\ were used to generate that document). \cite{Blei2003}
\ac{lda} assumes a generative process where, for each word in the document, the algorithm selects a distribution over topics, selects a topic, and then selects a vocabulary term. \cite{Blei2003}
By picking a distribution over topics, multiple possible topics can be blended into a single document.
Reversing this generative process is significantly more difficult because the topic distributions are unknown; this is what the ``hidden model'' or ``latent model'' refers to.

%------------------------------------------------

\ac{lda} can best be understood through its generative process.
Given the set of distributions as input, generating the corpus topics is a probabilistic process.
Taking the variables $\theta_{d,k}$ (topic proportion for topic $k$ in document $d$), $\beta_{1:k}$ (topic $k$), $z_{d,n}$ (topic assignment for word $n$ in document $d$), and $w_{d,n}$ (the $n^{th}$ word in document $d$), \ac{lda} calculates the posterior probability in Equation~\ref{eq:posterior}. \cite{Blei2012}

\begin{equation}
p(\beta_{1:K}, \theta_{1:D},z_{1:D} | w_{1:D}) = \frac{\beta_{1:K},
\theta_{1:D},z_{1:D}, w_{1:D}}{w_{1:D}}
\label{eq:posterior}
\end{equation}

%------------------------------------------------

\noindent
Using a variety of probabilistic methods to calculate or approximate the denominator (ie.\ evidence), \ac{lda} results in a usable set of vocabulary frequency distributions or topics.
Specifically, Gibbs Sampling, a variety of Bayesian Inference, is used to approximate the \ac{lda} posterior probability.~\cite{Blei2003}

%------------------------------------------------

\subsection{Related Work}
\label{sec:related-work}

%------------------------------------------------

The approach presented in this paper, specifically the inference of course concepts from catalog descriptions through \ac{lda} in order to evaluate and compare university departments, is novel.
However it is based on related work.

%------------------------------------------------

This work makes use of existing prerequisite relationships within departments.
After inferring novel topics for each course, the conceptual overlap between prerequisite courses can be computed.
However, similar approaches attempt to infer novel prerequisite relationships between courses.
Yang et al.\ employ four distinct techniques to map courses into a conceptual space and then learn prerequisite relationships between similar courses.~\cite{Yang2015}
Two of their conceptual mapping techniques generate latent features, which have the downside of not being human-readable as in \ac{lda}.
The remaining two techniques generate human-readable topics, but rely either on an outside source (Wikipedia) or simply represent concepts as the vocabulary of the document.
The benefit of \ac{lda} as an information retrieval tool is its ability to generate pseudo human-readable topics while acting in a fully unsupervised manner on a single, large data set.

%------------------------------------------------

Additional work has been done in the area of course description retrieval.
Our approach naively targets a set number of universities and tools web scrapers specifically to their computer science departments.
Effland et al.\ introduces a robust web crawler system to automatically search for, identify, and extract course descriptions from disparate locations on the Internet.~\cite{Effland2015}
Application of similar technology in this work was considered, and would greatly improve the scale of the analyzed data.


%------------------------------------------------

