\section{Future Work}
\label{sec:future-work}

%------------------------------------------------

One of the major benefits and weaknesses of \ac{lda} is its unsupervised nature.
Beneficially, it allows for the extraction of information from an entirely unknown dataset.
In this context, this flexibility allows its application to any number of diverse academic departments.
However, the main drawback of this characteristic is the lack of categorical information for the inferred topics.
While a topic can be understood via manual inspection of its terms, \ac{lda} offers no single comprehensive label to summarize it.
A possible solution to this problem might involve a meta-analysis of inferred topics.
Each topic could be classified as one of a number of learning outcomes based on its composition and weighting in a description.
Mapping the learned topics onto a standardized framework of learning outcomes~\cite{krathwohl2002} would allow for immediate integration of the extracted course concepts into existing academic evaluative frameworks based on learning outcome literature.

%------------------------------------------------

As educators who often have deficient resources to improve their pedagogy~\cite{Brown2013} look towards online or virtual mechanisms to support them, the system we have designed can be very useful.
Based on ideas discussed by Brown and K\"{o}lling~\cite{Brown2013}, one potential we see for future work is the integration of our system with an existing virtual community of CS educators, or the creation of community features around the system we have designed.
For instance, we can make it easier for educators to share or request resources from others or to learn more about why certain content is or is not covered in specific courses.
As more educators provide data to the system, the quality of results will benefit as well.

%------------------------------------------------

We also foresee that in the future we will be able to combine course related data with specific course assignments (through learning-management system (LMS) data), thereby providing a better picture of the kinds of experiences students can hope to receive in any given course.
This combination of data will also allow a better examination of the effects of different teaching strategies on students' learning~\cite{sanderson2000,kay2000,barker2004}.
For instance, we will be able to better understand the role of pedagogical techniques, such as problem-based and service-learning, on student outcomes.
By aligning with LMS data, we will also be able to learn more about how students perform on different assignments related to a specific competency or course content.
In the future we also plan to combine this data with data about student demographics and that can help provide a better picture of performance across gender and race~\cite{katz2006,sahami2010}.
This can be useful in designing more supportive pedagogical elements.
