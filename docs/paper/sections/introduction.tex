\section{Introduction}
\label{sec:introduction}

%------------------------------------------------

Computer ScIENCE EDUcation is an increasingly important field of growth at many universities.
As departments grow and change, it becomes necessary to automate the comparison and evaluation processes.
However, much of the published data about departments is non-standard, natural language text, not easy to process automatically.
There are many parties impacted by the lack of up to date, automatic, and simple to understand information about the characteristics of universities across the country.
Prospective college students and their parents seek out information on college courses to compare curricula in a meaningful way, based on content, in order to find their best fit.
A typical approach to this task is an information gathering and subsequent program comparison process duplicated many thousands of times across the population of rising college freshmen.
Aggregating the course description data to a central location and automating the comparison process saves time and effort.
Additionally, accrediting bodies (\eg\ ABET) typically require a department to cover a given set of standardized topics as a criterion for evaluation.
The accreditation process can take up to 18 months to complete~\cite{ABET2015}; automating the departmental evaluation process would greatly reduce time spent measuring a CS department's coverage of a specific set of areas.

%------------------------------------------------

Programs of study at institutions of higher education can be represented as a chain composed of the courses required to complete a degree.
These component courses in turn are composed of the topics or concepts they are intended to cover.
Evaluation of the courses within a particular program is necessary for the evaluation of an overall academic curriculum.
Analyzing the structure of a program's prerequisite chain, for example, requires an understanding of each constituent course and any overlap of covered topics between courses and their prerequisites.
Additionally, inter-institutional curricular comparison requires an aggregate evaluation of the courses within each institution's program.
However, comparing and evaluating different courses requires expert knowledge in the relevant field.
No two courses can be measured for similarity based only on inherent, measurable properties.
A domain expert is required to inspect the description of the courses and determine their conceptual overlap.

%------------------------------------------------

Automating the information retrieval process to identify core concepts covered in any particular course removes the need for a domain expert.
By analyzing course descriptions from a corpus spanning fields and institutions, topic modeling can provide a method to generate a statistical representation of core course concepts.
Specifically, unsupervised latent variable models present a method of identifying the core concepts (\ie\ topics) covered in a course.
This introduces the possibility of applications in automated course and program evaluation methods.
The form of topic modeling employed in this work is \acf{lda}.~\cite{Blei2003}

%------------------------------------------------

The overall goal of this work is the development of a system to digest large quantities of university course information, specifically academic course descriptions, and to process and ultimately generate interactive descriptions of the core concepts covered within institutional programs as illustrated by inferred topics.
Learned topics will be presented in a web-based application allowing inspection from multiple perspectives.

%------------------------------------------------

