\documentclass[10pt, compress]{beamer}

\usetheme{m}

\usepackage{booktabs}
\usepackage[scale=2]{ccicons}
\usepackage{minted}
\usepackage{graphicx}
\usepackage[backend=bibtex]{biblatex}

\usepgfplotslibrary{dateplot}

\usemintedstyle{trac}

\title{Unsupervised academic curricula evaluation through topic modeling}
\subtitle{}
\date{\today}
\author[jrouly@gmu.edu]{Jean Michel Rouly}
\institute[Department of Computer Science]{George Mason University}

\bibliography{../bibliography/bibliography.bib}

\begin{document}

\maketitle


\begin{frame}{What to expect from this talk}
  \begin{itemize}[<+- | alert@+>]
    \item Overview
    \item Background
    \item Data
    \item \textbf{Trajectory}
  \end{itemize}
\end{frame}


\section{Overview}


\begin{frame}{Motivation}

  Evaluating a department on its conceptual coverage involves\ldots
  \vfill
  \onslide<+->
  \begin{description}[<+- | alert@+>]
    \item[Relative Standing] comparing against other, similar departments
    \item[Absolute Performance] benchmark against standardized expectations
      (published by the ACM)
  \end{description}
  \vfill
  \uncover<+->{Both of these require data about the topics covered in a
    course.}

\end{frame}


\begin{frame}{Motivation}
  \begin{block}{What we've got:}
    \onslide<+->
    \begin{enumerate}[<+- | alert@+>]
      \item Widely available university course description data.
      \item Descriptions detail what concepts are taught in a course.
      \item \textbf{Human-readable} descriptions require manual inspection.
    \end{enumerate}
  \end{block}

  \vfill

  \uncover<5->{\begin{block}{So what to do?}\end{block}}

\end{frame}


\begin{frame}{Research Goal}

  \vfill
  Through the application of probabilistic machine learning methods,
  specifically LDA topic modeling, a corpus of unstructured course
  descriptions can be digested and mined for topics. In this scenario,
  each topic represents a core concept covered by the courses.
  \vfill
  A research framework will be constructed to read data from the
  Internet, digest into a common internal format, pipeline into an LDA
  topic model, and ultimately visualize in an interactive manner.
  \vfill
  Ultimately the automatically discovered topics can be used in end-user
  university evaluation processes.
  \vfill
\end{frame}


\begin{frame}{Research Goal}

  \begin{block}{In other words}
    Build a tool that automatically\ldots
    \onslide<+->
    \begin{itemize}[<+- | alert@+>]
      \item infers topics from a collection of course descriptions
      \item computes comparisons between departments
      \item evaluates departments on their concepts
    \end{itemize}
  \end{block}

\end{frame}


\section{Background}


\begin{frame}{Topic Modeling}
  Attempts to discover the abstract \alert{topics} of a dataset.

  \begin{block}{Topics}
    A \alert{topic} is a probabilistic distribution over terms, roughly
    representing a concept taught in a course.
  \end{block}
\end{frame}


\begin{frame}{Latent Dirichlet Allocation}
  \begin{block}{Overview}
    \alert{Latent Dirichlet Allocation (LDA)} is a form of \emph{Latent
    Variable Modeling} that can infer topics from within a document.
    \vfill
    LDA takes a generative approach to latent variable modeling, assuming
    the topics occur in some proportion within each document.
  \end{block}
\end{frame}


\section{Data}


\begin{frame}{University Dataset}
  \begin{table}
    \begin{tabular}{lcl}
      \toprule
      University & Course Count & Web \\
      \midrule
      American University & 32 & \href{http://american.edu}{american.edu} \\
      George Mason University & 145 & \href{http://gmu.edu}{gmu.edu} \\
      Kansas State University & 83 & \href{http://ksu.edu}{ksu.edu} \\
      Louisiana State University & 59 & \href{http://lsu.edu}{lsu.edu} \\
      Portland State University & 190 & \href{http://pdx.edu}{pdx.edu} \\
      Rensselaer Polytechnic Institute & 61 & \href{http://rpi.edu}{rpi.edu} \\
      University of South Carolina & 64 & \href{http://sc.edu}{sc.edu} \\
      Stanford University & 69 & \href{http://stanford.edu}{stanford.edu} \\
      University of Utah & 142 & \href{http://utah.edu}{utah.edu} \\
      University of Tennessee, Knoxville & 29 & \href{http://utk.edu}{utk.edu} \\
      \midrule
      ACM Exemplar Courses & 68 & --- \\
      \bottomrule
    \end{tabular}
    \caption{University course descriptions}
  \end{table}
\end{frame}


\begin{frame}{Sample Course Descriptions}
  \begin{block}{Raw Course Description}
    Capstone course focusing on design and successful implementation of
    major software project, encompassing broad spectrum of knowledge and
    skills, developed by team of students. Requires final exhibition to
    faculty-industry panel.
  \end{block}

  \begin{block}{Cleaned course description}
    capston focus design success implement major softwar project encompass
    broad spectrum knowledg skill develop team student requir final exhibit
    faculti industri panel
  \end{block}
\end{frame}


\section{Trajectory}


\begin{frame}{Trajectory}
  \textbf{Trajectory} is a tool that automatically ingests course
  description data from the Internet and presents an accessible interface
  for departmental evaluation.

  \begin{table}
    \begin{tabular}{ll}
      \toprule
      Lines of Python & 3193 \\
      Lines of Java & 631 \\
      Lines of HTML/CSS/JS & 1828 \\
      Lines of JSON & 3219 \\
      Lines of Bash & 165 \\
      \midrule
      Size on disk & 6.7M \\
      \bottomrule
    \end{tabular}
    \caption{Code statistics}
  \end{table}

\end{frame}


\begin{frame}
  \frametitle{Trajectory}
  Four primary modules:

  \begin{description}
    \item[Scrape] web-scrape online university catalogs
    \item[Import/Export] pass structured data between Learn and Scrape
    \item[Learn] estimate LDA topic model on data
    \item[Web] visualization tool
  \end{description}

  Underneath the entire system is a structured relational database layer.
\end{frame}


\begin{frame}
  \frametitle{Trajectory / Web}
  \begin{description}
    \item[Browse] collected data by university or department
    \item[Understand] courses through inferred topics
    \item[Analyze] conceptual overlap in prerequisite chains
    \item[Compare] departments based on conceptual composition
    \item[Evaluate] departments against ACM benchmarks
  \end{description}
\end{frame}


\plain{Live Demo \\\small wish me luck!}


\begin{frame}{Summary}

  Try out \textbf{Trajectory} online at

  \begin{center}\url{trajectory.staging.rouly.net}\end{center}

  Get the source of this presentation and the \textbf{Trajectory} project
  at

  \begin{center}\url{github.com/jrouly/trajectory}\end{center}

  \vfill

  \textbf{Trajectory} is licensed under the
  \href{http://www.apache.org/licenses/LICENSE-2.0}{Apache version 2.0
  license}.

\end{frame}


\begin{frame}
  \frametitle{Future Goals}
  \begin{itemize}
    \item Learning Outcomes meta-analysis
    \item Alternative methods
    \item Topic summarization
  \end{itemize}
\end{frame}



\begin{frame}{Credits}
  Co-authors:
  \begin{itemize}
    \item Huzefa Rangwala
    \item Aditya Johri
  \end{itemize}

  Presentation theme:
  \begin{itemize}\item\url{github.com/matze/mtheme}\end{itemize}
\end{frame}


\plain{Questions?}


\end{document}
