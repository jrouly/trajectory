At a high level, courses can be compared by the core concepts they address.
By extrapolating, programs of study are assumed to be the union over the set
of all concepts covered in the composite courses. Comparing these courses
or programs requires expert knowledge. There is no simple inherent or
natural relationship between two separate courses in distinct domains;
however, an expert in the domains would perhaps argue for a link via common
conceptual topics.

Through the application of  probabilistic machine learning methods,
specifically \ac{lda} topic modeling, a corpus of unstructured course
syllabi can be digested and mined for topics. In this scenario, each topic
represents a core concept covered by the courses. Knowledge of the topics
covered in courses as set out by course syllabi informs the design of
prerequisite chains and, more generally, the academic program built on
these courses.

The primary goal of this work is the development of a system to digest
large quantities of data, specifically academic course syllabi, and to
process and ultimately generate interactive descriptions of the topics
covered within institutional programs as illustrated by core course
concepts.

A secondary goal of this work is the identification of an unsupervised
means of generating labels for discovered topics. A primary weakness in
\ac{lda} and unsupervised learning methods in general is an inability to
name or otherwise identify the probability distributions generated as
topics.
