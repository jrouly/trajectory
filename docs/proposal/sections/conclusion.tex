Programs of study in higher education differ widely between departments and
universities. In a perfect world, the same program of study at two
institutions would provide students with the same core concepts. Indeed,
this is all too often not the case. Because of these discrepencies, program
accreditation and evaluation methodologies are employed with the goal of
standardizing the contents of a program of study. However, this process
generally requires manual inspection by a domain expert to extract
information from large quantities of course syllabi. Automating the
digestion and processing of these syllabi will greatly reduce the time and
effort required. Topic Modeling presents a statistical machine learning
method to extract the hidden topics behind a document set, in this case the
core concepts covered by a course syllabus. \acf{lda} results in a feasible
breakdown of textual syllabi into component concepts. More work will be
required to fully utilize the power of this tool, but the results
highlighted in this study show great promise.
