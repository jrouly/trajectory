Programs of study at institutions of higher education are a chain composed
of the courses required to complete a degree. These component courses in
turn are composed of the topics or concepts they are intended to cover.
Evaluation of the courses within a particular program is a key process in
the evaluation of an overall academic curriculum. Analyzing the structure
of a program's prerequisite chain, for example, requires an understanding
of each constituent course and the overlap, if present, of covered topics
between courses and their prerequisites. Additionally, inter-institutional
curricular comparison requires an aggregate evaluation of the courses
within each institution's relevant program. However, comparing and
evaluating different courses requires expert knowledge in the relevant
field. Placing, for example, two courses, one in data mining and one in
database theory, under evaluation may not yield any inherent similarity
unless a domain expert can identify the concepts expected from each class.

Automating the information retrieval process to identify core concepts
covered in any particular course removes the need for a domain expert. By
analyzing course syllabi from a corpus spanning fields and institutions,
topic modeling can provide a method to generate a semantic representation
of core course concepts. Specifically, unsupervised latent variable models
present a method of identifying the core concepts (ie.\ topics) covered in
a course. This introduces the possibility of applications in automated
course and program evaluation methods.
